\documentclass{article}

\usepackage[margin=1in]{geometry}
\usepackage{microtype}
\usepackage{fancyhdr}
\usepackage{amsmath}
\usepackage{mathtools}

\usepackage{mathpartir}

\pagestyle{fancy}
\fancyhf{}
\lhead{CS 225: Programming Languages}
\rhead{UVM, Spring 2018}

\setlength\arraycolsep{0pt}

\begin{document}
\thispagestyle{fancy}
\begin{center}
 \Large{Homework 1} \\ \ \\ \normalsize{Due: Friday, Jan 26, 11:59pm}
\end{center}

\subsubsection*{Preface}

Discussing high-level approaches to homework problems with your peers is
encouraged. You must include at the top of your assignment a Collaboration
Statement which declares any other people with whom you discussed homework
problems. For example:
\begin{quote}\em
  Collaboration Statement: I discussed problems 1 and 3 with Jamie Smith. I
  discussed problem 2 with one of the TAs. I discussed problem 4 with a
  personal tutor.
\end{quote}
If you did not discuss the assignment with anyone, you still must declare:
\begin{quote}\em
  Collaboration Statement: I did not discuss homework problems with anyone.
\end{quote}
Copying answers or doing the work for another student is not allowed.

Assignment problems which refer to ``Exercise X'' or ``Figure Y'' are referring
to those found in the Types and Programming Languages textbook.

\subsubsection*{Submitting}

Prepare your assignment as either handwritten or using LaTeX. \textbf{I will
not accept homework assignments written in Word, Google Docs, or using any
other text processing software.} Handwritten assignments must be written neatly
or they will receive a 0 grade. Submit the assignment either (1) via scanned
pdf email to me: David.Darais@uvm.edu with ``CS 225 HW1'' in the subject line,
or (2) placed under my office door (Votey~319) at any hour before the deadline.

\subsubsection*{Problem 1 (15 points)}

Recall the definition for the divides relation:
\[ \text{divides} \coloneqq \{ \langle n,m \rangle  \mid \exists o\ \textit{s.t.}\ n \times o = m \} \]
Prove formally---and in as much detail as possible---that the divides relation
is transitive, that is:
\[ \text{forall}\ n\ m\ o, \text{if}\ n\ \text{divides}\ m\ \text{and}\ m\ \text{divides}\ o,\ \text{then}\ n\ \text{divides}\ o \]
You may assume basic algebraic arithmetic facts like $n+n=2n$ and $2(n+n) =
2n+2n$. Use the example proof of reflexivity given in class as a guide for the
level of detail you should strive for.

\subsubsection*{Problem 2 (10 points)}

Consider the set of boolean arithmetic terms $\mathcal{T}$ and metafunctions leaves (new) and depth (from Definition 3.3.2):
\[ \begin{array}{l}
   t \in \mathcal{T} \Coloneqq T \mid F \mid \texttt{if}\ t\ \texttt{then}\ t\ \texttt{else}\ t
   \\
   \\ \text{leaves}(T) \coloneqq 1
   \\ \text{leaves}(F) \coloneqq 1
   \\ \text{leaves}(\texttt{if}\ t_1\ \texttt{then}\ t_2\ \texttt{else}\ t_3) 
      \coloneqq \text{leaves}(t_1) + \text{leaves}(t_2) + \text{leaves}(t_3)
   \\
   \\ \text{depth}(T) \coloneqq 1
   \\ \text{depth}(F) \coloneqq 1
   \\ \text{depth}(\texttt{if}\ t_1\ \texttt{then}\ t_2\ \texttt{else}\ t_3) 
      \coloneqq \text{max}(\text{depth}(t_1),\text{depth}(t_2),\text{depth}(t_3)) + 1
   \end{array}
\]
Define some term $t \in \mathcal{T}$ such that $\text{leaves}(t) = 7$ and $\text{depth}(t) = 3$.

\subsubsection*{Problem 3 (25 points)}

Either prove by structural induction that $\text{leaves}(t)$ always produces an
odd number, or give a counter-example which shows $\text{leaves}(t)$ can produce
an even number.

\subsubsection*{Problem 4 (15 points)}

Draw a derivation tree which justifies the following relationship:
\begin{gather*} 
  \texttt{if}
   \ (\texttt{if}
     \ (\texttt{if}\ F\ \texttt{then}\ F\ \texttt{else}\ T)
     \ \texttt{then}\ T
     \ \texttt{else}\ F)
   \ \texttt{then}\ T
   \ \texttt{else}\ F 
   \\ \longrightarrow
   \\ 
  \texttt{if}
   \ (\texttt{if}
     \ T
     \ \texttt{then}\ T
     \ \texttt{else}\ F)
   \ \texttt{then}\ T
   \ \texttt{else}\ F 
\end{gather*}

\subsubsection*{Problem 5 (30 points)}

Consider the extended small-step semantics described in Exercise 3.5.16 which
explicitly generates the value ``\texttt{wrong}'' in place of where the
semantics from Figure 3-2 gets stuck.
\begin{enumerate}
\item
  Design a big-step semantics $t \Downarrow a$ (similar to 3.5.17) which is
  equivalent to this small-step semantics.
\item
  Prove that your new big-step semantics implies the small-step semantics which
    generates ``\texttt{wrong}'', that is, prove $\textit{if}\;\;t \Downarrow
    a\;\;\textit{then}\;\;t \longrightarrow^* a$. This proof need not be as
    detailed as your answer to Problem~1, but still must be a convincing formal
    proof.
\end{enumerate}
You should use the following syntactic categories for terms $t$, numeric values
$nv$, values $v$, and answers $a$:
\[ \begin{array}{rcrcl}
     t {}&{} \in {}&{} \mathcal{T} {}&{} \Coloneqq {}&{} T \mid F \mid \texttt{if}\ t\ \texttt{then}\ t\ \texttt{else}\ t
  \\   {}&{}     {}&{}             {}&{} \mid      {}&{} 0 \mid \texttt{succ}\ t \mid \texttt{pred}\ t \mid \texttt{iszero}\ t
  \\   {}&{}     {}&{}             {}&{} \mid      {}&{} \texttt{wrong}
  \\ 
  \\  nv {}&{} \in {}&{} \mathcal{NV} {}&{} \Coloneqq {}&{} 0 \mid \texttt{succ}\ nv
  \\  v  {}&{} \in {}&{} \mathcal{V}  {}&{} \Coloneqq {}&{} T \mid F \mid nv
  \\  a  {}&{} \in {}&{} \mathcal{A}  {}&{} \Coloneqq {}&{} v \mid \texttt{wrong}
  \
   \end{array}
\]


\subsubsection*{Problem 6 (5 points)}

Approximately how many hours did you spend working on this assignment?

\end{document}
